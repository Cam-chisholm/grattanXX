\begin{overview}
\raggedright

The report for Grattan Institute \textit{Widening gaps: What NAPLAN tells us about student progress} seeks to measure and compare relative student progress on the National Assessment Program -- Literacy and Numeracy (NAPLAN) test in a way that is robust, easy to interpret, and comparable across different groups of students. It analyses student-level data to identify some of the factors associated with higher or lower rates of progress, and to quantify the degree of these associations. The analysis does not attempt to quantify the causal impact of these factors, and should not be interpreted as such.

Every year since 2008, the NAPLAN test has been administered Australia-wide to nearly all students in Years 3, 5, 7, and 9. This means that students who were in Year 3 in either 2008 or 2009 have now taken the NAPLAN test across each of the test-taking years. This makes it possible to track how much students have progressed (as measured by NAPLAN) over a significant proportion of their time spent at school.

This technical report includes four technical appendices to \textit{Widening gaps}. \Cref{chap1} describes the rationale and conceptual framework behind creating a new frame of reference to interpret NAPLAN results. \Cref{chap2} describes the data used in the analysis, and discusses some of the data issues. \Cref{chap3} outlines the technical detail behind the methodology to convert NAPLAN scale scores to \textit{equivalent year levels}. Finally, \Cref{chap4} explains the approach used to track the progress of students from Year 3 to Year 9, using Victorian linked data.

\end{overview}